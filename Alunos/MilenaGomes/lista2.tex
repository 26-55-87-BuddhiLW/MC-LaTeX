\documentclass[12pt]{abntex2}
\usepackage{times}
\usepackage[T1]{fontenc}		% Selecao de codigos de fonte.
\usepackage[utf8]{inputenc}		% Codificacao do documento (conversão automática dos acentos)
\usepackage{enumitem} %enumerate com (a), (b), (c); (I), (II), etc
\usepackage{amsmath}
\usepackage{esint}
\usepackage{graphicx}
\usepackage[brazil]{babel}
\pagestyle{empty}
\usepackage[table]{xcolor}
%% TITLE
%% -----------------------------
\title{Lista 2}
\author{Milena Gomes da Silva\\ %% Student name
  Minicurso \LaTeX \\ %% Code and course name
  EEL - USP
}

\date{Junho, 2020} %% Change "\today" by another date manually
%% -----------------------------
%% -----------------------------
height=0.65\textwidth
%% %%%%%%%%%%%%%%%%%%%%%%%%%
\begin{document}

\setlength{\droptitle}{-5em}
\maketitle
\thispagestyle{empty}
% --------------------------
% Start here
% --------------------------

% %%%%%%%%%%%%%%%%%%%

\section*{Respostas}

\subsection*{1.(a)}
	\begin{equation}
		\begin{cases}
			\frac{{\mathrm d} x}{{\mathrm d}t} = \sigma(y-x) \\
			\frac{{\mathrm d} y}{{\mathrm d}t} = x(\rho-z)-y \\
			\frac{{\mathrm d} z}{{\mathrm d}t} = xy-\beta z
		\end{cases}
	\end{equation}

\subsection*{1.(b)}
\begin{equation}
F(k)=\frac{1}{2\pi}\int_{-\infty}^{\infty}s(x){e^{-ikx}}\mathrm{d}x
\end{equation}
\subsection*{2.(a)}

\begin{table}[!htb]

	\IBGEtab{%
		\caption{\centering{Parâmetros das peças de cobre}}%
		\label{tab:ibge}
	}{%

		\begin{tabular}{ccccccc}
			\toprule
			Geometria & A(m^2) & V(m^3) & L_s$\mathrm{(m)}$ & \rho$\mathrm{(kg/m^3)}$ & c_p$\mathrm{(J/kgK)}$ & k_s$\mathrm{(W/mK)}$\\
			\midrule \midrule
			Placa & 0,0378 & 0,0002 & 0,0054 & 8930 & 386 & 398 \\
			\midrule
			Cilindro & 0,0280 & 0,0003 & 0,0107 & 8930 & 386 & 398 \\
			\midrule
			Esfera & 0,0082 & 0,00007 & 0,0085 & 8930 & 386 & 398 \\
			\bottomrule
		\end{tabular}%
	}{%
		\fonte{A autora.}%
	}

\end{table}
\newpage
\subsection*{2.(b)}
 \begin{table}[htb]
 \centering
 \caption{\centering{Parâmetros das peças de cobre}}
\begin{tabular}{|ccccccc|}

\hline
\rowcolor[rgb]{0.2,0.9,0.8}
\hline
 Geometria & A(m^2) & V(m^3) & L_s$\mathrm{(m)}$ & \rho$\mathrm{(kg/m^3)}$ & c_p$\mathrm{(J/kgK)}$ & k_s$\mathrm{(W/mK)}$ \\ \hline
 \hline
 \rowcolor[rgb]{0.7,0.9,0.8}
 \hline
Placa & 0,0378 & 0,0002 & 0,0054 & 8930 & 386 & 398 \\
\hline
\rowcolor[rgb]{0.7,0.9,0.8}
\hline
Cilindro & 0,0280 & 0,0003 & 0,0107 & 8930 & 386 & 398 \\
\hline
\rowcolor[rgb]{0.7,0.9,0.8}
\hline
Esfera & 0,0082 & 0,00007 & 0,0085 & 8930 & 386 & 398 \\
\hline
\end{tabular}
\fonte{A autora.}
\end{table}
\newpage
\begin{figure}[!htb]
	\subsection*{(3)}
	\begin{center}
	\caption{\label{fig:2}Girassóis, Van Gogh}

	\includegraphics[width=0.85\textwidth, height=0.65\textheight]{C:/Users/nili1/OneDrive/Favoritos/Documentos/LabEEL/LaTeX/Imagens/girassol}

\legend{Fonte: National Gallery, 1888.}
	\end{center}
\end{figure}
\end{document}
%%% Local Variables:
%%% mode: latex
%%% TeX-master: t
%%% End:
