\documentclass{beamer}

%% Pacotes
\usepackage[utf8]{inputenc}              %%Formatação de acentos
\usepackage{xcolor}               %%Definição de cores
\usepackage{calligra}

\usepackage[framemethod=tikz]{mdframed} %%Estilização dos slides
\usepackage{tcolorbox}                   %%Mudar ambientes de blocos
\usepackage[activate={true,nocompatibility},final,tracking=true,kerning=true,spacing=true,factor=1100,stretch=10,shrink=10,expansion]{microtype}
%% Melhorar caligrafia
\SetTracking{encoding={*}, shape=sc}{0} %Caligrafia de small capitals (smallcaps)
\usepackage{pifont} %Dingbats (simbolos do itemize)
\usepackage[alf]{abntex2cite}
{\usebackgroundtemplate{\includegraphics[width=1\paperwidth, height=1\paperheight]{../Imagens/bb}}
%% Tema da apresentação
\color{red}
\usetheme{CambridgeUS}


\mode<presentation>


\title[Introdução ao \LaTeX]{\bfseries\Huge{Clube do Livro}}
\author[Milena Gomes]{\bfseries\textcolor{white}{Milena Gomes da Silva} \\
\text{\scriptsize{\textcolor{white}{milena.gomes.silva@usp.br}}}}

\date[17.06.2020]



%%%%%% ~~~~~~~~~ %%%%%%%%

%%%%%%%%%%% Começo do Corpo do Documento %%%%%%%%%%%%%%%%

%%%%%% ~~~~~~~~~ %%%%%%%%


\begin{document}

%% Imagem de fundo para a apresentação,

{\usebackgroundtemplate{\includegraphics[width=1\paperwidth, height=1\paperheight]{../Imagens/biblioteca}}

%% Alterar o diretório ''../Imagens/TP.jpg'' para mudar de imagem


%%%%%% Primeira slide, utilizando as variáveis author, date etc.
  \begin{frame}
    \titlepage
  \end{frame}
}



\begin{frame}

\section{O que nos motiva?}
\frametitle{{\bfseries{O que nos motiva?}}}
	\begin{tcolorbox}[colback=red!5!white, colframe=red!65!black,
		title={\sc{\bf{O que lemos?}}}]
		No Clube do Livro, nós lemos desde romance clássico do século XIX, passando por sci-fi dos anos 80 e chegando até os livros do nosso século.
	\end{tcolorbox}
\end{frame}

\begin{frame}
\section{O que nos motiva?}
\frametitle{{\bfseries{O que nos motiva?}}}
\setbeamercovered{invisible}
\begin{itemize}[<+->]
	\item[{\textcolor{red!90!black}{\ding{109}}}]{Ler é interpretar o mundo em que vivemos.} \cite{Paulo}
	\item[{\textcolor{red!90!black}{\ding{109}}}]{O personagem lia, e ao ler sentia-se limpo e livre de qualquer doença.}
	\cite{VHM}
\end{itemize}

\end{frame}

\begin{frame}
	\section{Como funciona o Clube do Livro?}
	\frametitle{\bfseries{Como funciona o Clube do Livro?}}
	\setbeamercovered{invisible}
	\begin{itemize}[<+->]
		\item[{\textcolor{red!90!black}{\ding{109}}}]{Grupo no Telegram}
		\item[{\textcolor{red!90!black}{\ding{109}}}]{Escolha do livro por enquete}
		\item[{\textcolor{red!90!black}{\ding{109}}}]{Número determinado de capítulos por semana}
		\item[{\textcolor{red!90!black}{\ding{109}}}]{Discussão sobre o livro uma vez por semana}
	\end{itemize}

\end{frame}

\begin{frame}
  \section{Livros já lidos}
  \frametitle{\bfseries{Livros já lidos}}
  \begin{itemize}
  \item[{\textcolor{red!90!black}{\ding{111}}}]{A Máquina de Fazer Espanhóis} \\
  \includegraphics[width=0.2\linewidth]{../Imagens/vhm}
  \item[{\textcolor{red!90!black}{\ding{111}}}]{Jane Eyre}
  \item[{\textcolor{red!90!black}{\ding{111}}}]{O Filho de Mil Homens}
  \item[{\textcolor{red!90!black}{\ding{111}}}]{O Nome da Rosa}
  \end{itemize}
\end{frame}

\begin{frame}
	\section{Livros já lidos}
	\frametitle{\bfseries{Livros já lidos}}
	\begin{itemize}
		\item[{\textcolor{red!90!black}{\ding{111}}}]{A Máquina de Fazer Espanhóis}
		\item[{\textcolor{red!90!black}{\ding{111}}}]{Jane Eyre}\\
		\includegraphics[width=0.2\linewidth]{../Imagens/je}
		\item[{\textcolor{red!90!black}{\ding{111}}}]{O Filho de Mil Homens}
		\item[{\textcolor{red!90!black}{\ding{111}}}]{O Nome da Rosa}
	\end{itemize}
\end{frame}

\begin{frame}
	\section{Livros já lidos}
	\frametitle{\bfseries{Livros já lidos}}
	\begin{itemize}
		\item[{\textcolor{red!90!black}{\ding{111}}}]{A Máquina de Fazer Espanhóis}
		\item[{\textcolor{red!90!black}{\ding{111}}}]{Jane Eyre}
		\item[{\textcolor{red!90!black}{\ding{111}}}]{O Filho de Mil Homens}\\
		\includegraphics[width=0.2\linewidth]{../Imagens/vhm2}
		\item[{\textcolor{red!90!black}{\ding{111}}}]{O Nome da Rosa}
	\end{itemize}
\end{frame}

\begin{frame}
	\section{Livros já lidos}
	\frametitle{\bfseries{Livros já lidos}}
	\begin{itemize}
		\item[{\textcolor{red!90!black}{\ding{111}}}]{A Máquina de Fazer Espanhóis}
		\item[{\textcolor{red!90!black}{\ding{111}}}]{Jane Eyre}
		\item[{\textcolor{red!90!black}{\ding{111}}}]{O Filho de Mil Homens}
		\item[{\textcolor{red!90!black}{\ding{111}}}]{O Nome da Rosa} \\

		\includegraphics[width=0.2\linewidth]{../Imagens/nm}
	\end{itemize}
\end{frame}

\begin{frame}
		\section{Nossa próxima leitura}
	  	\frametitle{\bfseries{Nossa próxima leitura}}
		\begin{tcolorbox}[colback=red!5!white, colframe=red!65!black,
			title={\sc{\bf{Os Robôs}}}]
		\begin{minipage}{.5\linewidth}
			Um dos mais eminentes cientistas de Solária
			fora encontrado brutalmente assassinado e  só um Terrestre poderia desvendar o desconcertante  e sombrio mistério. E assim, mesmo contrariados,  os Solarianos pediram o auxílio de um Terrestre.
			\cite{Is}
		\end{minipage}
		\begin{minipage}{0.5\linewidth}
		\begin{figure}
			\includegraphics[width=0.4\linewidth]{../Imagens/r}
		\end{figure}
		\end{minipage}
		\end{tcolorbox}
\end{frame}

\begin{frame}
\frametitle{\bfseries{Referências}}
\bibliography{ref}
\end{frame}

\end{document}
%%% Local Variables:
%%% mode: latex
%%% TeX-master: t
%%% End:
