%%% Local Variables:
%%% mode: latex
%%% TeX-master: t
%%% End:


%%% Template originaly created by Karol Kozioł (mail@karol-koziol.net) and modified for ShareLaTeX use

\documentclass[a4paper,11pt, dvipdfmx]{abntex2}

\usepackage[T1]{fontenc}
\usepackage[utf8]{inputenc}
\usepackage{graphicx}
\usepackage{xcolor}
\usepackage{movie15}

\renewcommand\familydefault{\sfdefault}
\usepackage{tgheros}
% \usepackage{Consolatas}
\usepackage{yfonts}

\usepackage{amsmath,amssymb,amsthm,textcomp}

\setlist[2]{noitemsep} % sets the itemsep and parsep for all level two lists to 0

\setenumerate{noitemsep} % sets no itemsep for enumerate lists only
% \begin{enumerate}[noitemsep] % sets no itemsep for just this list
%   % ...
% \end{enumerate}


\usepackage{multicol}
\usepackage{tikz}

\usepackage[margin=1in]{geometry}
\geometry{left=25mm,right=25mm,%
  bindingoffset=0mm, top=20mm,bottom=20mm}


\linespread{1.1}

\newcommand{\linia}{\rule{\linewidth}{0.5pt}}

% custom theorems if needed
\newtheoremstyle{mytheor}
{1ex}{1ex}{\normalfont}{0pt}{\scshape}{.}{1ex}
{{\thmname{#1 }}{\thmnumber{#2}}{\thmnote{ (#3)}}}

\theoremstyle{mytheor}
\newtheorem{defi}{Definition}

% my own titles
\makeatletter
\renewcommand{\maketitle}{
  \begin{center}
    \vspace{4ex}
    { {\fontsize{40}{40}\selectfont \textfrak{\@title}} }
    \vspace{2ex}
    \\
    \linia\\
    \@author \hfill \@date
    \vspace{6ex}
  \end{center}
}
\makeatother
%%%

% custom footers and headers
\usepackage{fancyhdr}
\pagestyle{fancy}
\lhead{}
\chead{}
\rhead{Entrega até 14/05}
\lfoot{Lista \textnumero{} 1}
\cfoot{}
\rfoot{Página \thepage}
\renewcommand{\headrulewidth}{0.5pt}
\renewcommand{\footrulewidth}{0.5pt}
%

%%% ----------%%%----------%%%----------%%%----------%%%

\usepackage{hyperref}

{%Muda a cor do Sumário, pois são todo links.
  \hypersetup{
    colorlinks=true,
    citecolor= violet,
    linkcolor=black!85,
    filecolor=magenta,
    urlcolor=cyan,
  }

  %%% ----------%%%----------%%%----------%%%----------%%%
  %% Animação
  %\usepackage[loop]{animate}
  \usepackage{animate}
  %\usepackage{media9}
  \usepackage{graphicx}


  \begin{document}

  \title{Lis:ta 2}

  \author{\indent  Prof.: Pedro G. Branquinho, EEL-USP. \\
    Orient.: Katia C. G. Candioto.}

  \date{10/05/20 - 17/05/20}

  \maketitle

  \section*{Problema 1}
\textbf{2.(a) Reescreva o seguinte sistema de equações,} as quais
descrevem a convecção do ar atmosférico. Dessas equações, deriva-se os
atratores de Lorentz, conhecido pela sua forma de borboleta. Daonde
sai o famigero conceito de ``Efeito Borboleta''\footnote{\url{https://en.wikipedia.org/wiki/Lorenz_system}}.

\begin{equation}
  \label{eq:boboleta}
  \begin{cases}
    \frac{\mathrm{d}x}{\mathrm{d}t} = \sigma (y - x) \\
    \frac{\mathrm{d}y}{\mathrm{d}t} = x(\rho - z) - y\\
    \frac{\mathrm{d}z}{\mathrm{d}t} = x y - \beta z
  \end{cases}
\end{equation}

% (Usando o pacote movie15, é possível colocar gifs nos documentos)


% \begin{center}

%   \caption{Gif 1 - Atrator de Lorenz}

%   \animategraphics[scale=0.2, autoplay, poster]{12}{./gif/animate_}{0}{150}
%   \animategraphics{12}{gif/animate_}{0}{150}
%   % \includemovie[autoplay]{./gif/download.gif}
%   \legend{Fonte: http://docs.juliaplots.org/latest/}
%   \end{center}
  \clearpage

\noident\textbf{1.(a) Reescreva a seguinte fórmula,} transformada de
Fourier na forma complexa,

\begin{equation}
  F(k) = \frac{1}{2 \pi} \int_{-\infty}^{\infty}{s(x)e^{-ikx} \mathrm{d}x}
\end{equation}

\section*{Problema 2}

\textbf{Faça duas tabelas, uma na formatação IBGE, e outra em qualquer outra
formatação, com os mesmos elementos.}



\section*{Problema 3}

\textbf{Coloque uma imagem no documento, centralizada, com 85\% da
  largura da página, e 65\% da altura página (imagens portrato ficarão
  numa resolução boa para essa configuração).}


\end{document}

%%% Local Variables:
%%% mode: latex
%%% TeX-master: t
%%% End:
