x%%% Local Variables:
%%% mode: latex
%%% TeX-master: t
%%% End:

\documentclass{standalone}
\usepackage{pgfgantt}
\usepackage{xcolor}
\renewcommand{\rmdefault}{ptm}
\usepackage[scaled=0.92]{helvet}
% \usepackage{lmodern}
\usepackage{courier}
\normalfont % in case the EC fonts aren't available
\usepackage[T1]{fontenc}
\parskip=2pt\parindent 0pt


\definecolor{orchid}{RGB}{218, 112, 214}
\definecolor{purple1}{RGB}{144, 5, 133}
\definecolor{orange1}{RGB}{255, 165, 0}

\definecolor{brisa1}{RGB}{150, 150, 150}
\definecolor{brisa2}{RGB}{128, 128, 128}

\definecolor{mesa1}{RGB}{188,218,221}
\definecolor{mesa2}{RGB}{250, 250, 250}

\definecolor{louça1}{RGB}{138, 234, 145}
\definecolor{louça2}{RGB}{192,192,192}

\definecolor{sid1}{RGB}{255, 127, 0}
\definecolor{sid2}{RGB}{0, 128, 255}

\definecolor{branqs1}{RGB}{19, 221, 120, 1}
\definecolor{branqs2}{RGB}{20, 222, 222}



\begin{document}

\begin{figure}[!bth]

  \begin{center}

    \begin{ganttchart}[
      vgrid={*{2}{black!40, dashed}, *{1}{purple, thick}},
      hgrid={1*{grey}},
      title/.append style=%
      {fill=blue!12, rounded corners=0.4mm, drop shadow},
      title label font=\color{black!70}\bfseries,
      title height=0.8,
      title top shift=0.2,
      % title left shift=0.2,
      % title right shift=-0.1,
      chart element start border=right]
      {1}{31}
      \gantttitle[title/.style={draw=black,
        fill=orchid}]{\LARGE{\textbf{\fontfamily{put}\selectfont Março}}}{31} \\

      \gantttitlecalendar{day} \\[grid]

      %%%%%%%%%%%%%%%
      %%%%%%%%%%%%%%%

      \ganttgroup[group/.style={draw=black, outer
        color=branqs1, inner color=branqs2!50}]{\LARGE{\textsc{{Completo}}}}{0}{31} \\

      \ganttbar[bar/.style={draw=black, outer
        color=brisa2, inner color=brisa1}]{\large{Finalização aula 1
          e 2}}{1}{15} \ganttbar[bar/.style={draw=black, outer
        color=brisa2, inner color=brisa1}]{}{25}{30}\\

      \ganttbar[bar/.style={draw=black, outer
        color=mesa1, inner color=mesa2}]{\large{Material Online}}{15}{30} \\

      \ganttbar[bar/.style={draw=black, outer
        color=black, inner color=black}]{\large{Relatório}}{20}{31} \ganttbar[bar/.style={draw=black, outer
        color=black, inner color=black}]{}{25}{28}\\

      \ganttbar[bar/.style={draw=black, outer
        color=louça1, inner color=louça2}]{\large{Elaboração de Exercícios}}{20}{31}

      %%%%%%%%%%%%%%%
      %%%%%%%%%%%%%%%
    \end{ganttchart}
  \end{center}
\end{figure}


\vspace{1cm}
\begin{large}
  {Nota 1: a``Finalização das aulas 1 e 2'' são alterações mínimas do fundo
    de apresentação; procura de erros ortográficos; mudanças revisadas
    pela orientadora.

    Nota 2: a secção ``Material Online'' representa o período disposto
    para criar uma turma no GoogleClassroom, e organizar o material de
    forma a ser utilizado sem necessidade de um tutor.

    Nota 3: no ``Relatório'' e ``Elaboração de Exercícios'' ainda não foi
    terminada a revisão do relatório - faltando atualizar o cronograma
    diante da paralização do COVID-19. E, o tempo desprendido para criar
    exercícios que possam ser feitos à distância e entrege.}
\end{large}



\end{document}

%%% Local Variables:
%%% mode: latex
%%% TeX-master: t
%%% End:
