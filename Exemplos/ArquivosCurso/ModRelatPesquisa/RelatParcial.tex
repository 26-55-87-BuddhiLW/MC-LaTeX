%% abtex2-modelo-relatorio-tecnico.tex, v-1.9.7 laurocesar
%% Copyright 2012-2018 by abnTeX2 group at http://www.abntex.net.br/
%%
%% This work may be distributed and/or modified under the
%% conditions of the LaTeX Project Public License, either version 1.3
%% of this license or (at your option) any later version.
%% The latest version of this license is in
%%   http://www.latex-project.org/lppl.txt
%% and version 1.3 or later is part of all distributions of LaTeX
%% version 2005/12/01 or later.
%%
%% This work has the LPPL maintenance status `maintained'.
%%
%% The Current Maintainer of this work is the abnTeX2 team, led
%% by Lauro César Araujo. Further information are available on
%% http://www.abntex.net.br/
%%
%% This work consists of the files abntex2-modelo-relatorio-tecnico.tex,
%% abntex2-modelo-include-comandos and abntex2-modelo-references.bib
%%

% ------------------------------------------------------------------------
% ------------------------------------------------------------------------
% abnTeX2: Modelo de Relatório Técnico/Acadêmico em conformidade com
% ABNT NBR 10719:2015 Informação e documentação - Relatório técnico e/ou
% científico - Apresentação
% ------------------------------------------------------------------------
% ------------------------------------------------------------------------

\documentclass[
	% -- opções da classe memoir --
	12pt,				% tamanho da fonte
	openright,			% capítulos começam em pág ímpar (insere página vazia caso preciso)
	oneside,			% para impressão em recto e verso. Oposto a oneside
	a4paper,			% tamanho do papel.
	% -- opções da classe abntex2 --
	%chapter=TITLE,		% títulos de capítulos convertidos em letras maiúsculas
	%section=TITLE,		% títulos de seções convertidos em letras maiúsculas
	%subsection=TITLE,	% títulos de subseções convertidos em letras maiúsculas
	%subsubsection=TITLE,% títulos de subsubseções convertidos em letras maiúsculas
	% -- opções do pacote babel --
	english,			% idioma adicional para hifenização
	french,				% idioma adicional para hifenização
	spanish,			% idioma adicional para hifenização
	brazil,				% o último idioma é o principal do documento
	]{abntex2}


% ---
% PACOTES
% ---

% ---
% Pacotes fundamentais
% ---
\usepackage{lmodern}			% Usa a fonte Latin Modern
\usepackage[T1]{fontenc}		% Selecao de codigos de fonte.
\usepackage[utf8]{inputenc}		% Codificacao do documento (conversão automática dos acentos)
\usepackage{indentfirst}		% Indenta o primeiro parágrafo de cada seção.
\usepackage{color}				% Controle das cores
\usepackage{graphicx}			% Inclusão de gráficos
\usepackage{microtype} 			% para melhorias de justificação
% ---

% ---
% Pacotes de citações
% ---
\usepackage[brazilian,hyperpageref]{backref}	 % Paginas com as citações na bibl
\usepackage[alf]{abntex2cite}	% Citações padrão ABNT


% ---
% Pacotes adicionais, usados no anexo do modelo de folha de identificação
% ---
%\usepackage{multicol}
%\usepackage{multirow}
% ---

% ---
% Pacotes adicionais, usados apenas no âmbito do Modelo Canônico do abnteX2
% ---
%\usepackage{lipsum}				% para geração de dummy text
% ---


% ---
% CONFIGURAÇÕES DE PACOTES
% ---

% ---
% Configurações do pacote backref
% Usado sem a opção hyperpageref de backref
\renewcommand{\backrefpagesname}{Citado na(s) página(s):~}
% Texto padrão antes do número das páginas
\renewcommand{\backref}{}
% Define os textos da citação
\renewcommand*{\backrefalt}[4]{
	\ifcase #1 %
		Nenhuma citação no texto.%
	\or
		Citado na página #2.%
	\else
		Citado #1 vezes nas páginas #2.%
	\fi}%
% ---

% ---
% Informações de dados para CAPA e FOLHA DE ROSTO
% ---
\titulo{Minicurso de \LaTeX}
\autor{Aluno: Pedro G. Branquinho \\ Orientadora: Katia Cristiane
  Gandolpho Candioto}
\local{Lorena, São Paulo}
\data{13 de Fevereiro de 2020}
\instituicao{%
  Universidade de São Paulo - USP \\
  Escola de Engenharia de Loreena
  \par
  LabEEL - DEMAR
  \par
  Projeto de Bolsa PUB}
\tipotrabalho{Relatório técnico}
% O preambulo deve conter o tipo do trabalho, o objetivo,
% o nome da instituição e a área de concentração
%\preambulo{Relato da elaboração e progresso do minicurso de \LaTeX}
% ---

% ---
% Configurações de aparência do PDF final

% alterando o aspecto da cor azul
\definecolor{blue}{RGB}{41,5,195}

% informações do PDF
\makeatletter
\hypersetup{
     	%pagebackref=true,
		pdftitle={\@title},
		pdfauthor={\@author},
    	pdfsubject={\imprimirpreambulo},
	    pdfcreator={LaTeX with abnTeX2},
		pdfkeywords={abnt}{latex}{abntex}{abntex2}{relatório técnico},
		colorlinks=true,       		% false: boxed links; true: colored links
    	linkcolor=blue,          	% color of internal links
    	citecolor=blue,        		% color of links to bibliography
    	filecolor=magenta,      		% color of file links
		urlcolor=blue,
		bookmarksdepth=4
}
\makeatother
% ---

% ---
% Espaçamentos entre linhas e parágrafos
% ---

% O tamanho do parágrafo é dado por:
\setlength{\parindent}{1.3cm}

% Controle do espaçamento entre um parágrafo e outro:
\setlength{\parskip}{0.2cm}  % tente também \onelineskip

% ---
% compila o indice
% ---
\makeindex
% ---

% ----
% Início do documento
% ----
\begin{document}

% Seleciona o idioma do documento (conforme pacotes do babel)
%\selectlanguage{english}
\selectlanguage{brazil}

% Retira espaço extra obsoleto entre as frases.
\frenchspacing

% ----------------------------------------------------------
% ELEMENTOS PRÉ-TEXTUAIS
% ----------------------------------------------------------
% \pretextual

% ---
% Capa
% ---
\imprimircapa
% ---


% ---
% Agradecimentos
% ---
% \begin{agradecimentos}
% O agradecimento principal é direcionado a Youssef Cherem, autor do
% \nameref{formulado-identificacao} (\autopageref{formulado-identificacao}).

% Os agradecimentos especiais são direcionados ao Centro de Pesquisa em
% Arquitetura da Informação\footnote{\url{http://www.cpai.unb.br/}} da Universidade de
% Brasília (CPAI), ao grupo de usuários
% \emph{latex-br}\footnote{\url{http://groups.google.com/group/latex-br}} e aos
% novos voluntários do grupo
% \emph{\abnTeX}\footnote{\url{http://groups.google.com/group/abntex2} e
% \url{http://www.abntex.net.br/}}~que contribuíram e que ainda
% contribuirão para a evolução do abn\TeX.

% \end{agradecimentos}
% ---

% ---
% RESUMO
% ---

% resumo na língua vernácula (obrigatório)
\setlength{\absparsep}{18pt} % ajusta o espaçamento dos parágrafos do resumo
\begin{resumo}
Com o intuito de capacitar os alunos da instituição EEL-USP a
utilizarem da ferramenta \LaTeX para tipografia, propôs-se um programa
de ensino, em formato de minicurso, com uso de apostila didática. O
\LaTeX, é uma linguagem markdown de formatação de documentos. E se especializa em reproduzir trabalhos científicos, bem
como ser fiemente reprodutível, quanto a seus documentos. O curso focará no
ensino da ferramenta, com pacotes específicos para formatação em
acordo com as normas ABNT, e os modelos canônicos do pacote. Também,
serão ministradas aulas sobre a formatação de apresentações, e posters.

 \noindent
 \textbf{Palavras-chaves}: minicurso. latex. modelos canônicos ABNT.
\end{resumo}
% ---


% ---
% inserir o sumario
% ---
\pdfbookmark[0]{\contentsname}{toc}
\tableofcontents*
% ---


% ----------------------------------------------------------
% ELEMENTOS TEXTUAIS
% ----------------------------------------------------------
\textual

% ----------------------------------------------------------
% Introdução (exemplo de capítulo sem numeração, mas presente no Sumário)
% ----------------------------------------------------------
\chapter[Introdução]{Introdução}
%\addcontentsline{toc}{chapter}{Introdução}

O \LaTeX, um sistema de produção textual computacional, surgiu para
sistematizar toda a tipografia de documentos por meio digital. É
reconhecido, largamente, em formatação de linguagem matemática, e
produção documental científicas; docucumentos multilinguísticos; e,
especialmente, produções longas e complexas, como teses
\cite{ignat2005}

Por meio de uma sistematização localizada, por via de códigos, há uma
regularidade, previsibilidade, e reproduvidade superior em relação a
programas do tipo WYSIWYG, ``What You See Is What You Get'' - o que se
visualiza é o que se reproduz. Um exemplo desse paradigma é o software
Word, o qual é propriedade privada da Microsoft.

Além do mais, por ser um programa amplamente desenvolvido pela
comunidade como linguagem open source, ganha-se qualidade na
reutilização e evolução linguística dos usuários-desenvolvedores
\cite{goossens1994}. Pois, um arquivo, template, pacote, ou classe,
pode ser reutilizado, uma vez criado, para a resolução de problemas
recorrentes à comunidade. Desta forma, advém pacotes coo o abnTeX,
desenvolvido pela equipe \abnTeX, CPAI - UnB, no Centro de Pesquisa em
Arquitetura de Informação. O pacote se compraz de ferramentas, e
modelos canônicos feitos estritamente sob às normas ABNT, os quais
podem ser utilizados e adaptados a todas intituições brasileiras que
siga as normas.


\section{Objetivo}

Por meio de minicurso profissionalizante, objetiva-se eninar alunos da
EEL-USP a usarem a linguagem markup, \LaTeX. Pela utilização de
pacotes de formatação abnTeX2, o aluno precisa apenas focar no texto,
e material da pesquisa - pois a formatação é automática. Assim,
aumenta-se as chances de melhorar a qualidade de produção de trabalhos
acadêmicos, que é base da filosofia da linguagem.

O objetivo do projeto é de ministrar quatro aulas de um minicurso de
\LaTeX, para alunos de gradução da EEL, explanando sobre a filosofia,
metodologia, e técnicas de produção de relatórios, teses, artigos, e
cartazes gráficos, como posters.

\chapter{Revisão Bibliográfica}

\section{O \LaTeX}

\LaTeX foca em separa a formatação do texto de seu conteúdo. Desta
forma, o usuário concetra-se exclusivamente em seu conteúdo, em um
estágio. E, na formatação de sua aparência, em outro. Assim, ganha-se
qualidade de produção. Bem como total autonomia sob o documento, pois
a programação da disposição do documento depende apenas do usuário, e
pode ser indefinidamente estenso. O sistema tipográfico de \LaTeX - o
\TeX - já chegou a ser considerado o sistema digital de
tipografia mais sofisiticado que existe \cite{haralambous2007}.

O \LaTeX, tecnicamente, é a junção do sistema de tipografia \TeX,
inventado por Donald Knuth, para tipografia de alto nível
\cite{knuth1986}; com os poderosos macros que facilitam a estensão do programa \TeX, a qual damos o nome de
\LaTeX. O \LaTeX foi iniciamente desenvolvido por Leslie Lamport, com
seus pacotes fundamentais de formatação \cite{lamport1994} O \LaTeX,
por conseguinte, não é somente uma linguagem de tipografia de alto
nível, mas também um conjunto de macros para facilitar a tipografia em
si. Qualifica-se, assim, como um sistema de preparação de documentos;
uma linguagem markup de domínio específico.

\section{Classe Canônica ABNT de produção científica}

Documentos sob os requisitos das normas ABNT (Associação Brasileira de Normas
Técnicas) para elaboração de documentos técnicos e cintíficos
brasileiros, como artigos científicos, relatórios técnicos trabalgos
acadêmicos como teses, dissertações, projetos de pesquisa e outros
documentos do gênero \cite{abntex2012} é ao que se chama classe
canônica ABNT.

\begin{citacao}
Os documentos indicados tratam-se de “Modelos Canônicos”, ou seja,
de modelos que não são específicos a nenhuma universidade ou instituição, mas
que implementam exclusivamente os requisitos das normas da ABNT, Associação
Brasileira de Normas Técnicas. \cite[Cap. 1]{araujoclasse}
\end{citacao}

As normas as quais prescrevem o modelo canônico são:

\begin{itemize}
  \item \textbf{ABNT NBR 6022:2018:} Informação e documentação -
    Artigo em publicação periódica científica - Apresentação.
  \item \textbf{ABNT NBR 6023:2002:} Informação e documentação -
    Referência - Elaboração
  \item \textbf{ABNT NBR 6024:2012:} Informação e documentação -
    Numeração progressiva das seções de um documento - Apresentação
  \item \textbf{ABNT NBR 6027:2012:} Informação e documentação -
    Sumário - Apresentação
  \item \textbf{ABNT NBR 6028:2003:} Informação e documentação -
    Resumo - Apresentação
    \item \textbf{ABNT NBR 6029:2006:} Informação e documentação -
      Livros e folhetos - Apresentação
    \item \textbf{ABNT NBR 6034:2004:} Informação e documentação -
      Índice - Apresentação
    \item \textbf{ABNT NBR 10520:2002:} Informação e documentação -
      Citações
    \item \textbf{ABNT NBR 10719:2015:} Informação e documentação -
      Relatórios técnicos e/ou científico - Apresentação
    \item \textbf{ABNT NBR 14724:2011:} Informação e documentação -
      Trabalhos acadêmicos - Apresentação
    \item \textbf{ABNT NBR 15287:2011:} Informação e documentação -
      Projeto de pesquisa - Aprensetação
\end{itemize}

\section{abnTeX}

O pacote abnTeX foi criado para suprir as necessidades de
formataçõese, em padrão ABNT. E, por conseguinte, auxiliar o aumento
do nível de produção nacional. De acordo com o autor,

\begin{citacao}
Dentre as características de qualidade de trabalhos acadêmicos (teses, dissertações e
outros do gênero), de artigos científicos, de relatórios técnicos e de livros e folhetos,
ao lado da pertinência do tema e dos aspectos relativos ao conteúdo abordado no
trabalho, consta também o resultado da editoração final e as características de
forma e de estruturação dos documentos. Desse modo, a existência de um modelo
e de ferramentas que atendam às normas brasileiras de elaboração de trabalhos
acadêmicos, artigos científicos e relatórios técnicos propostas pela Associação
Brasileira de Normas Técnicas (ABNT) são recursos básicos para o aprimoramento
da qualidade geral dos trabalhos acadêmicos nacionais.

É com esse intuito que o abn\TeX2 é apresentado à comunidade acadêmica brasileira:
o de ser um instrumento de aperfeiçoamento da qualidade dos textos produzidos.
O abnTEX2 surge para se somar ao já vasto universo de ferramentas LATEX, porém
que é escasso em utilitários específicos para trabalhos brasileiros. \cite[2.1]{araujoclasse}
\end{citacao}

O pacote, para se tornar parte do corpo oficial de pacotes \LaTeX, foi
desenvolvido desde 2001, até 2013. E, hoje, é mantida pela comunidade
de software livre. Seu acesso à atual distribuição oficial pode ser
feita em \url{https://www.ctan.org/pkg/abntex2}. CTAN - Comprehensive TEX Archive Network - é o site
repositório dos pacotes ``oficiais'' do \LaTeX.

\section{Beamer}

Desenvolvido pela comunidade, Beamer não é a primeira, porém, o mais
utilizado pacote para produção de slides e apresentações. Seus
desenvolvedores, inciais, foram Louis Stuart, Till Tantau, Joseph
Wright, e Vedran Miletc \cite{tantau2010}. Por mais que estes sejam os principais
desenvolvedores, Beamer é um pacote livre e aberto, como o \LaTeX, em
si. Isto é, toda a comunidade usuária é também desenvolvedora do
pacote. Desta forma, é um pacote extensivamente trabalhado para
produção de apresentações, diponível em diversas formatações
canônicas.

\section{}

 \cite{baramidze2014} \cite{hwang1995}.

\chapter{Materiais e Métodos}

Para a preparação do curso, usou os seguintes perseguiu-se as
seguintes etapas,

\begin{enumerate}
\item Determinação da sala das aulas - sediação;
\item A disponibilidade de vagas para alunos;
\item Datas;
\item Seleção de tópicos que devem ser lecionados sobre a ferramenta
  \LaTeX;
\item Quais métodos, e datas, seriam empregados para divulgação do minicurso.
\end{enumerate}

\section{Sediação}
O curso foi idealizado para até quarenta alunos. Procurou-se uma sala
que os comportasse, bem como tivesse os devidos equipamentos
eletrõnicos para sediar apresentações



\section{Data}
Procurou-se fazer escolha de datas que maximizassem a aderência, e
notoriedade do curso.


\section{Elaboração do Minicurso}

As divisões do curso foram feitas, esquematicamente, da seguite maneira, para cada umas
das quatro aulas:

\begin{itemize}
\item Introdução, Filosofia, e Instalação do LaTeX - utilização de templates TeX.
\item Produção de Relatórios, Teses e Monografias, Sob Norma ABNT -
  pacote ABNTeX, imagens, tabelas, referências bibliográficas, e
  citações ABNT.
\item Controle e Modulação dos Parâmetros de Pacotes e Templates - produção de apresentações com pacote Beamer.
\item Produção de Templates - produção de posters, banners e certificados.  Entrega do certificado de proficiência em LaTeX.
\end{itemize}


\section{Divulgação}

Procurou-se determinar quais eram os canais mais populares, e com
maiores retornos de público, para se divulgar. Bem como alocar a divulgação em datas estratégias.





\chapter{Resultado}

\section{Sediação}
Resolveu-se por alocar-se na sala EF-15, nas instalações do DEMAR,
devido à capacidade de suportar até 40 alunos; haver a suporte à
projetores, cabos conectores e acesso à internet.

\section{Data}
De acordo com as datas do calendário USP, obtido na plataforma
Jupiterweb, as aulas começam uma semana antes do carnaval. E, são
demarcadas como semana de apresentação e recepção dos calouros. Assim,
decidiu-se, estrategicamente, alocar o início do minicurso para depois
do carnaval.

\section{Elaboração do Minicurso}
Durante o período de Agosto a Janeiro, desenvolveu-se as apresentações, e
uma apostila virtual, a qual contém material suplementar de estudo e
refência. A apostila será
disponibilizada por meio do repositório
\url{https://github.com/26-55-87-BuddhiLW/MC-LaTeX}, bem como
materiais de apoio, exemplos, e os modelos canônicos ABNT usados nas aulas.

Procurou-se, para a escolha dos tópicos, e partição das aulas,
entender qual eram as necessidades dos alunos, quanto a produção
acadêmicos. Ao mesmo tempo, o curso foi projetado de forma a maxizar o
aprendizado sistemático do assunto.  Determinou-se que, tanto a
formatação sob as normas ABNT, como a formatação de apresentações eram
essenciais.

Assim, deu-se enfoque nas segunda e terceira aulas nesses
tópicos exclusivamente, e uma introdução reforçada, alocada à primeira
aula,  na base teórica da programação \LaTeX. Por fim, na quarta aula,
mostra-se mais como modular as configurações de pacotes, o que reforça
os conhecimentos prévios, e expande a capacidade do aluno a fazer
documentos variados, como posters e cartões.

Com o software Krita, criou-se os planos de fundo que serão utilizados
pelo grupo para a apresentação das aulas. Para isso, utilizou-se logos, e
logomarcas, da USP, da EEL, e do grupo LabEEL.

\section{Divulgação}

No mês de fevereiro a partir do dia 17, utilizará-se de mídias sociais e de correspondência, como Facebook,
Whatsapp, e e-mail usp, para que os alunos sejam informados do curso,
bem como anunciar o período de inscrição. Escolhe-se o período de
divulgação com base nas datas de feriados e início de aulas.

Com o software Krita, criou-se uma arte conceptiva, para divulgação do
curso. Utilizou-se, para essa produção, do logo oficial do LabEEL.


\chapter{Conclusão Parcial}

O projeto, nesse interim, se compraz de fase de produção de materiais,
e planejamento. Considera-se ter sido efetivo o desenvolvimento do
planejamento, e coordenação física do projeto. As quatro apresentações já
foram produzidas.

Algumas atividades extras, como criação de arte para divulgação, e
modelos de planos de fundo padrão para todo o grupo da LabEEL que fará
apresentações no mesmo formato.

Não é possível, ainda, reportar conclusões e dados sobre o objetivo do
projeto. Pois, a ministração do minicurso ainda não ocorreu. Por conseguinte,
esses resultados serão reportados apenas no relatório final.


\bibliography{bib}


\end{document}

%%% Local Variables:
%%% mode: latex
%%% TeX-master: t
%%% End:
